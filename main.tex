\documentclass{article}

\begin{document}

\section{Company, Product}
Ericsson, Common Operation and Maintenance  (COM)

\subsection{Organization}

\begin{itemize}
\item The management was waterfall
\item The management should have a sense of agile development as well.
\item The bottom, where Amir was, tried to change variable
\item Ericsson is an old organization
\item Customer does not understand the way of working in Ericsson
\end{itemize}

\section{Definition of case study}

\subsection{Objective - what to achieve?}

Amir group has been using methodology SCRUM, did it work for them? Did
they adhere methodology?

\subsection{The case - what is studied?}

The software methodology developement, and does help, is it efficient?

\subsection{Theory - frame of reference?}

SCRUM - the theory, what does it promise. Project management
technique. Fast iterations - iterative and incremental,

Agile:
\begin{itemize}
\item Putting most resposibility into teams. Teams decide how to
  do. Decision
\item Short cycle, adjust on requirements.
\end{itemize}

SCRUM, one paraticular implementation of agile thinking.
\begin{itemize}
\item using a board.
\end{itemize}

\subsection{Research questions - what to know?}

Does the theory provide what it promises?

\subsection{Methods - how to collect data?}

Interview Amir.

\subsection{Selection Strategy - where to seek data}

Again Amir


\section{Interview questions}

\begin{description}
\item[Describe the way you work.]
  \begin{itemize}
  \item Product owner - technical knowledge about system,
    architecture, prototyping new features, technical guy. Other
    organization, SCRUM prescribed. he has knowledge.
  \item Project manager - relation to customer, talking about what they
    need. Tight with customer. Comes back with requirements and
    feautres. Gives the list of requirements to the Product
    owner. Not part of team. He has the money. Takes the final
    decision.
  \item The PO - assessment of feasiblty . Sometimes PO consults the
    team.
  \item Backlog -
  \item How many teams? usually 3-5 PO.
  \item Customers came up features. ``When failed login three times,
    block him'', equivalent to the requirement.
  \item PO + PM : Technical description of the feature. ``Counter,
    database, row, different access, some unblock user,'' PO , make
    subrequirements: ``Be able to log login failed. ''

  \item SCRUM MASTER, takes the requirements to the team. The team
    decides if they want to do this. Development team breaks up the
    requirements into user stories to different tasks, which could be
    the implementation of functions, that tests that need to be
    defined.
  \item Sprint planning: breaking requirements, user story, are we
    correct. invite PO to clarify. Something is not feasible. Write
    the requierments in a template in a document.

  \end{itemize}
\item[What parts of the SCRUM METHODOLOGY did you use?]
\item[If he in any way felt limited in this way of working? (SCRUM)]
\item[How much communication was there formally and informally?]
\item[How much of the overall systems did you have to understand?]
\item[Did you do testing of your own code?]
\item[Did other people other people test your code?]
\item[How quickligy could you respond to changes in the requirements?
  (SCRUM offer fast changes)]
\item[Any negative sides of SCRUM?]
\item[Any positive sides with SCRUM?]

\item[Anything other methodology that was used?]
  \begin{itemize}
  \item They used code review.
  \item All levels of testing.
  \end{itemize}
\item[How did the methodology account for changes in the project,
  e.g. downsize the personel?]
\item[What tools did you use?]
  \begin{itemize}
  \item Git
  \item gerrit
  \item jenkins
  \end{itemize}
\item[Anecdotes]
  \begin{itemize}
  \item He tried to change a variable, didnt make it.
  \end{itemize}
\end{description}

\section{Analysis}

\subsection{Conclusions}


\end{document}

%%% Local Variables:
%%% mode: latex
%%% TeX-master: t
%%% End:
